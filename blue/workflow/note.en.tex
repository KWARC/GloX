\documentclass[12pt]{bluenote}
\usepackage[show]{ed}
% \usepackage{listings}
% \lstset{basicstyle=\sf,columns=fullflexible}
\usepackage[hyperref,backend=bibtex,style=alphabetic]{biblatex}

\addbibresource{../bibs/kwarcpubs.bib}
\addbibresource{../bibs/extpubs.bib}
\addbibresource{../bibs/kwarccrossrefs.bib}
\addbibresource{../bibs/extcrossrefs.bib}
\usepackage{hyperref}
\usepackage{cleveref}

\title{The \FAUstairs Glossary Extraction and Curation Process}
\author{Michael Kohlhase\\
  Computer Science, FAU Erlangen-N\"urnberg\\
  \url{http://kwarc.info/kohlhase}}

\def\defemph#1{\textbf{#1}}
\def\FAUstairs{\textsf{FAUstairs}\xspace}
\blueProject{\FAUstairs}
\def\GloX{\textsf{GloX}\xspace}
\def\studon{\textsf{StudOn}\xspace}
\def\campo{\textsf{Campo}\xspace}
\def\DIP{\textsf{DIP}\xspace}
\def\FloDown{\textsf{FloDown}\xspace}

\begin{document}
\maketitle
\begin{abstract}
  We describe the initial ideas for the \FAUstairs glossary extraction and curation
  process and the workflows and tooling we envision to supports it.

  This blue note is (supposed to be) a living document that describes the current state of
  the discussion, to serve as an implementation guide and initial documentation for the
  \GloX tool ecosystem.
\end{abstract}
\tableofcontents\newpage

\section{Introduction}\label{sec:intro}

A Central part of the \FAUstairs project (``Formative Assessment for Universities:
Strategic Application of Innovative Methods to Raise Study Success Rates'' see
\url{https://faustairs.fau.de}) is the development and curation of a \defemph{domain
  model} -- i.e. a set of key concepts and their definitions -- for large portions of the
courses at FAU (and the development of added-value services on top of that to establish
formative assessment).

In the following we describe the information sources, the glossary extraction and curation
workflows and the \GloX tool ecosystem.

\section{Information Sources and Stakeholders}\label{sec:sources}

The main sources of information for the \FAUstairs domain model are the
following\ednote{MK: I am sure there are more, need to extend}:
\begin{enumerate}
\item the module descriptions in \campo: \url{https://campo.fau.de}
\item the course infrastructure and curriculum data on \studon: \url{https://studon.fau.de}
\item the course materials of the instructors. 
\end{enumerate}
The first two are available via the \DIP system, a centralized infrastructure and data
store for synchronization of the FAU learning administration systems provided by the FAU
RRZE.

The stakeholders in the \GloX process are\ednote{MK: there must be more; extend}
\begin{enumerate}
\item The \defemph{degree programs} represented by the program directors (the faculty
  member formally in charge), the program coordinators and maybe the study advisors.
\item The \defemph{departments} that host the degree programs, represented by their
  speakers and the department manager.
\item The instructors of the mandatory courses of a degree program; here we include the
  persons who organize the tutorials, homework assignments, and (summative) assessments.
\item The \defemph{\FAUstairs GloXers} -- three pairs of knowledge representation and
  domain specialists tasked with the \GloX process.
\end{enumerate}

\section{Workflow}\label{sec:main}

The \GloX workflow will consist of two large steps glossary extraction and glossary
curation, which we will sketch out in the following: 

\subsection{Glossary Extraction}\label{sec:glox}

In this step we examine the information sources from \cref{sec:sources} for
glossary-relevant information and export it into a curation format (most probably
\FloDown).

The relevant steps are
\begin{enumerate}
\item \defemph{Concept Identification}: The domain specialists identify the key concepts
  in the information source
\item \defemph{Concept Annotation}: The concepts are annotated with
  \begin{enumerate}
  \item a \defemph{symbol} name (a system identifier), the concept in the source serves as the
    default verbalization. 
  \item (optionally) known \defemph{synonyms}, and 
  \item a \defemph{definition} (rigorous) or \defemph{concept documentation} (less
    rigorous description).
  \end{enumerate}
\item \defemph{Translation}\ednote{MK: do we want to do this?; I think it will be
    necessary at least for Math, INF and the natural sciences}: Where the scientific
  discourse is international, the concept names are standardized to their English
  versions.
\end{enumerate}

\subsection{Domain Model Curation}\label{sec:gloc}

In this step we collect all the available glossaries, aggregate them into a coherent
domain model. The relevant steps are
\begin{enumerate}
\item \defemph{Collection}: The glossaries are collected and systematically organized into
  a modular collection, most probably managed and served by MathHub.info.
\item \defemph{Annotation}: The definitions are further annotated with term references
  into the joint domain model by the GloXers.
\item \defemph{Aggregation}: this is mainly a de-duplication step, which identifies
  possible duplicate concepts (probably by their definitions and/or usage patterns).
\item \defemph{Canonicalization}: The domain model is compared against the disciplinary
  learning ontologies, etc.
\end{enumerate}

\section{The \GloX Tool Ecosystem}\label{sec:glox}

\begin{sloppypar}
\printbibliography
\end{sloppypar}
\end{document}

%%% Local Variables: 
%%% mode: latex
%%% TeX-master: t
%%% End: 

% LocalWords:  GloXers
